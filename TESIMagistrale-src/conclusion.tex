\chapter{Conclusions and Outlook}
In this project, we have discussed some applications of modern on-shell techniques for the computation of loop amplitudes. We have reviewed the spinor helicity formalism and factorization properties of helicity amplitudes. Using analytic and unitarity properties, we have observed interesting relations which create a connection between the desired amplitude and the results in the case of fewer legs and loops.\\
We have applied these methods in order to study the all-plus sector of Yang-Mills at one-loop and two-loop levels. These examples show that on-shell methods could represent a powerful and promising possibility to overcome the theoretical bottleneck in the computation of scattering amplitudes at higher multiplicity and level of precision.\\

We have considered the Higgs Effective Field Theory in which the Higgs boson is directly coupled to QCD via an effective vertex in the top mass limit. The computation of a two-loop amplitude which describes the interaction between the Higgs boson and four gluons could have important phenomenological applications. We have worked in a model in which the effective interaction is splitted into the sum of two term: the self dual and the anti-self dual sectors. In this model, Dixon, Glover and Khoze \cite{Dixon_2004} found interesting MHV rules with surprising simple tree-level structures.\\
In order to take advantage from the simplicity of the tree-level MHV $\phi$+gluon amplitudes, we have applied on-shell methods for the computation of the cut-constructible part of the two-loop $\phi$ plus four gluon amplitude. This has represented an interesting example of cut calculations in order to extract new informations. We have obtained the discontinuity of the amplitude (\ref{discnew}) and we have checked the pole structure and the collinear limit factorisation.\\

The effects of the dimensional regularisation can produce finite contributions without discontinuities. Cuts in four dimensions cannot investigate this missing information and we have to use an alternative approach in order to extract the full amplitude.\\
Recent computations of two-loop amplitudes combine the Feynman diagram approach with numerical sampling over finite fields. The analytic expression could be reconstructed from multiple numerical evaluations. The \textsc{FiniteFlow} framework \cite{Peraro_2019} can be used for functional reconstruction of amplitudes \cite{Peraro_2016} in ever-increasing number of two-loop corrections for phenomenological processes, for example \cite{Badger:2021ega, Badger:2021nhg}. The reconstruction of the amplitude with this approach should be more efficient if one first subtracts the cut-constructible part by isolating the missing information of the unitarity approach.\\
The complete reconstruction of the $\phi$ and four gluon amplitude in the all-plus sector represents a natural continuation of this project. In order to compute the full amplitude, a possible strategy is the use of the recent progress in the workflow based on finite fields. It represents a promising approach in order to overcome the theoretical bottleneck of the Feynman diagrams.\\
For what concerns the phenomenological applications in Higgs physics, we should also need the amplitude with $\phi^\dagger$ coupled with gluons. In the all-plus sector, the helicity amplitude shows a non-vanishing tree-level result, hence the computation should be more difficult than the self-dual counterpart. Again, the modern techniques based on finite fields could be useful to achieve the desired result for the anti-self dual Higgs.