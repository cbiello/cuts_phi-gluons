\chapter{Rational pieces (2L $\phi+4g^+$) from finite-field arithmetic [TO DO]}
\begin{equation}
	\tikzfeynmanset{ my2blob/.style={ shape=circle, typeset=$\bigcirc\bigcirc$,
draw=black, } }
\begin{tikzpicture}[baseline=(current bounding box.center)]
  \begin{feynman}
    \diagram [large, horizontal=b to c] {
           b [my2blob] -- [gluon] a [particle=\(1^+\)], %uso solo per distanziare i due blob, ma essendo bianchi verranno ricoperti
      b -- [gluon] w1 [particle=\(2^+\)],
      b  -- [gluon] d1 [particle=\(3^+\)],
      b -- [gluon] w2 [particle=\(4^+\)],
      b -- [scalar] d [ particle=\(\phi\)],
    };
  \end{feynman}
\end{tikzpicture}
\end{equation}

\begin{equation}
	\tikzfeynmanset{ my2blob/.style={ shape=circle, typeset=$\bigcirc\bigcirc$,
draw=black, } }
\begin{tikzpicture}[baseline=(current bounding box.center)]
  \begin{feynman}
    \diagram [small] {
           b [my2blob] -- [gluon] a [particle=\(1^+\)], %uso solo per distanziare i due blob, ma essendo bianchi verranno ricoperti
      b -- [gluon] w1 [particle=\(3^+\)],
      b  -- [gluon] d1 [particle=\(2^+\)],
      b -- [gluon] w2 [particle=\(4^+\)],
%      b -- [scalar] d [ particle=\(\phi\)],
    };
  \end{feynman}
\end{tikzpicture}
\end{equation}

\begin{equation}
	\tikzfeynmanset{ my2blob/.style={ shape=circle, typeset=$\bigcirc\bigcirc$,
draw=black, } }
\begin{tikzpicture}[baseline=(current bounding box.center)]
  \begin{feynman}
    \diagram [] {
           b [my2blob] -- [gluon] a [particle=\(1^+\)], %uso solo per distanziare i due blob, ma essendo bianchi verranno ricoperti
      b -- [gluon] w1 [particle=\(3^+\)],
      b  -- [gluon] d1 [particle=\(2^+\)],
      b -- [gluon] w2 [particle=\(5^+\)],
      b -- [gluon] d [ particle=\(4^+\)],
    };
  \end{feynman}
\end{tikzpicture}
\end{equation}