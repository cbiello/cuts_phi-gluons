\begin{abstract}
QCD corrections at two-loop order for the Higgs plus four gluon amplitude are fundamental ingredients to obtain the theoretical predictions at NNLO for the Higgs production with two jets. 
In order to capture these QCD effects, we can work in the large top mass limit and treat the Higgs as the real part of a complex scalar field $\phi$ coupled to the self-dual part of the gluon field strength. In this project, we apply on-shell techniques to compute the discontinuities of a two-loop $\phi$ plus four gluon amplitude which represents one of the two contributions to the corresponding Higgs amplitude.\\
Firstly, we review some on-shell methods to compute scattering amplitudes. We introduce the spinor-helicity formalism to ensure on-shell simplicity is manifest and summarise methods of recursion relations at tree-level and unitarity techniques at loop level.\\
The interaction between the $\phi$ field and gluons presents a vanishing tree-level amplitude in the all-plus sector, as observed in gauge theories. Led by this analogy, we start by understanding the structure of one-loop and two-loop amplitudes in the all-plus sector of Yang-Mills.\\
Then, after a summary of some known results about the interaction between the self-dual Higgs and all-plus gluons at tree and one-loop level, we describe our computation of the cut-constructible pieces for the two-loop amplitude which couples $\phi$ to four gluons in the all-plus configuration. After subtracting terms related to the universal pole structure, we obtain the finite remainder function written in terms of logarithms and dilogarithms. Finally, we check the factorisation properties in collinear limits.
%Lastly, we will try to extract the missing rational information from finite-field arithmetic.\\


%The two-loop amplitude coupling the Higgs with gluons will be the sum of the amplitudes with $\phi$ and $\phi^\dagger$, therefore the amplitude that we study in this project represents one of the two contributions to the Higgs amplitude in all-plus sector at two-loop level.\\
%Led by the previous analogy, we start by understanding the structure of one-loop and two-loop amplitudes in the all-plus sector of Yang-Mills: this represents an interesting background to learn the generalized unitarity methods in four dimensions and cuts in generic dimensions.\\
%For this amplitude, considerable simplifications arise because all the cuts in four dimentions of the one-loop all-plus amplitudes vanish.\\
%In order to extract the discontinuities, the unitarity condition requires to consider the sum of three different contributions: 
%\begin{enumerate}
%\item[(1)] we consider all the possibile double cuts involving a one-loop $\phi$+gluon amplitude and a tree level Yang-Mills amplitude;
%\item[(2)] we compute the double cuts which factorize in a product of a tree level $\phi$+gluon amplitude and a one-loop amplitude with only gluons;
%\item[(3)] we observe that the contributions from 3-particle cuts vanish in all-plus sector.
%\end{enumerate}

\end{abstract}
%\newpage
