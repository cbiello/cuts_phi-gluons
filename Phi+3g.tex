%Tutoraggio1
\documentclass{article}

\usepackage[english]{babel}
\usepackage[utf8]{inputenc}

\usepackage[margin=3cm]{geometry}

\usepackage{graphicx}
\graphicspath{{./pics/}}

\usepackage{amsmath}
\usepackage{amssymb}
\usepackage{mathtools}
\usepackage{cases}
\usepackage{physics}
%\usepackage{bbold} %caratteri speciali come matrice identità
\usepackage{graphicx}
\usepackage{wrapfig}
\usepackage{subfig}
\usepackage{array} %per centrare tabelle
\usepackage{array}
\usepackage{tabularx} 
\usepackage{longtable}
\usepackage{sidecap}
\usepackage{tikz}	%diagrammi di Feymann
\usepackage[compat=1.1.0]{tikz-feynman}
\usepackage{marginnote}
\usepackage[export]{adjustbox}
\usepackage{booktabs,siunitx}
\usepackage{adjustbox}
\usepackage{float}
\usepackage{afterpage}
\usepackage{csquotes}
\usepackage{multicol}
\usepackage[hidelinks,colorlinks]{hyperref}
\usepackage{caption}
\usepackage{hhline}
\usepackage{makecell}
\usepackage{dsfont}
\usepackage{tensor}
\usepackage[titletoc]{appendix}
\usepackage{enumitem}
\setenumerate[1]{label=\arabic*.}
\newcounter{ResumeEnumerate}
\usepackage{amsthm} %per le dimostrazioni
% Uncomment the following line to allow the usage of graphics (.png, .jpg)
%\usepackage{graphicx}

\newtheorem*{lem}{Lemma}
\newtheorem*{theorem}{Teorema}
\newtheorem*{mydef}{Definizione}

\hypersetup{
	colorlinks=true,
	linkcolor=teal,
	filecolor=teal,      
	urlcolor=teal,
	citecolor=teal,
}

\renewcommand\thefootnote{\textcolor{teal}{\arabic{footnote}}}
\hypersetup{
	colorlinks=true,
	linkcolor=teal,
	filecolor=teal,      
	urlcolor=teal,
	citecolor=teal,
}

\usepackage{xcolor}
\colorlet{eqlink}{teal}
\newcommand*{\SavedEqref}{}
\let\SavedEqref\eqref
\renewcommand*{\eqref}[1]{%
	\begingroup
	\hypersetup{
		linkcolor=eqlink,
		linkbordercolor=eqlink,
	}%
	\SavedEqref{#1}%
	\endgroup
}

\newenvironment{parteo}{\begin{quote} \captionsetup[figure]{font=footnotesize,labelfont=footnotesize} \footnotesize}{\end{quote}}
\newcommand{\quantities}[1]{%
	\begin{tabular}{@{}c@{}}\strut#1\strut\end{tabular}%	\\per andare a capo nelle tabelle
}

\newenvironment{result}{\begin{flushright}
		\vspace{-0.3cm}\small[}{] \end{flushright}}
	
\newcommand\numthis{\addtocounter{equation}{1}\tag{\theequation}}
	\numberwithin{equation}{section}
	
	\newcommand{\Ie}{\textit{i.e. }}
	\newcommand{\eg}{e.g. }
	\newcommand{\NB}{\textit{N.B.:}}
	\newcommand{\R}{\mathbb{R}}
	\newcommand{\Z}{\mathbb{Z}}
	\newcommand{\wrt}{w.r.t. }
	\newcommand{\rhs}{r.h.s. }
	\newcommand{\tabitem}{~~\llap{\textbullet}~~}
	
	\usepackage{afterpage}
	\newcommand\blankpage{%
		\null
		\thispagestyle{empty}%
		\addtocounter{page}{-1}%
		\newpage
	}

\markboth{$\phi+3g^+$ cc}{$\phi+3g^+$ cc}
\pagestyle{myheadings}

% Start the document
\begin{document}
\renewcommand{\abstractname}{\vspace{-\baselineskip}}
\newpage
%\newgeometry{margin=2.7cm}
\begin{center}
\vspace{0.5cm}
	\textbf{\Large $\phi+3g^+$}\\\vspace{0.1cm}
	\textbf{\large Cut-constructible part}
	\vspace{0.5cm}
\end{center}
{
	\hypersetup{linkcolor=teal,linktoc=page}
	%\tableofcontents
	\thispagestyle{empty}
}
\numberwithin{equation}{section}

\noindent We have four different double-cuts which can be divided into two sectors.
Computing the double cuts, we obtained the cut-constructible part of the amplitude,
\begin{align*}
A^{2L}_{cc}(\phi;1^+2^+3^+)=&\, A^{1L}(\phi;1^+,2^+,3^+)\sum_{\sigma\in \mathbb{Z}_3} d_{\sigma(i)} I_{4,\sigma(1)}^{1m}\,+A^{1L}(\phi;1^+,2^+,3^+)\sum_{\sigma\in \mathbb{Z}_3} c_{\sigma(1)} I_{3,\sigma(1)}^{2m}\\&-\frac{d_s-2}{6} A^{1L}(\phi;1^+2^+3^+)\, I_2(s_\phi)
\end{align*}
where
\begin{align*}
	d_{\sigma(1)}=-\frac{1}{2}s_{\phi \sigma(1)}s_{\phi\sigma(3)}, \hspace{0.5cm}c_{\sigma(1)}=2\frac{1}{2}(s_{\sigma(1)\phi}-s_\phi).
\end{align*}
Then the divergent part is
\begin{align*}
	\left[A^{2L}_{cc}(\phi;1^+,2^+,3^+)\right]_{IR+UV}=\left[-\frac{1}{\epsilon^2}\sum_{i=1}^3 \left(-s_{i,i+1}\right)^{-\epsilon}-\frac{d_s-2}{6\epsilon}\right].
\end{align*}
The remainder part comes from the finite contribution of one-mass boxes and from the bubble integral. We obtain
\begin{align*}
	\left[A^{2L}_{cc}(\phi;1^+2^+3^+)\right]_{finite}=A^{1L}(\phi;1^+2^+3^+)\left[2\,\text{Li}_2\left(1-\frac{s_\phi}{s_{12}}\right)+2\,\text{Li}_2\left(1-\frac{s_\phi}{s_{23}}\right)+2\,\text{Li}_2\left(1-\frac{s_\phi}{s_{31}}\right) \right.\\ \left.
+\frac{1}{2}\ln^2\frac{s_{12}}{s_{23}}+\frac{1}{2}\ln^2\frac{s_{23}}{s_{31}}+\frac{1}{2}\ln^2\frac{s_{31}}{s_{12}}+\frac{\pi^2}{2}-\frac{d_s-2}{6}\left(2+\ln\left(\frac{\mu_R^2}{-s_\phi}\right)\right)\right].
\end{align*}
As soon as possible, I will write the cut-constructible pieces in terms of one-loop MI.\\
In the following sections there are some details about the computation.
\section{Cuts with a 1L $\phi+g$ sub-amplitude}
\begin{tabularx}{\linewidth}{XX}
\vspace{-1cm}
\begin{equation}	\tag{\text{dcut A}} 
    \begin{aligned}
\tikzfeynmanset{ whiteblob/.style={ shape=circle, typeset=$1L$,
draw=black, } }
\tikzfeynmanset{ myblob/.style={ shape=circle, typeset=$tree$,
draw=black, } }
\begin{tikzpicture}
  \begin{feynman}
    \diagram [scale=0.9, horizontal=b to c] {
      b [myblob] --  [white] db -- [white] c [whiteblob], %uso solo per distanziare i due blob, ma essendo bianchi verranno ricoperti
      b -- [white] ds -- [white] c,
      a [particle=\(2^+\)] -- [gluon] b
        -- [gluon, half left, out=60, in=120, rmomentum=\(\ell_1\)] c
        -- [gluon, half left, in=120, out=60, rmomentum=\(\ell_2\)] b ,
      d1 [particle=\(1^+\)] -- [gluon] b,
      d3 [particle=\(3^+\)]-- [gluon] b,
      c -- [scalar] d [particle=\(\phi\)],
    };

    %% Find the midpoint, which is halfway between b and c.
    \coordinate (midpoint) at ($(b)!0.5!(c)$);
    %% Draw a line starting 2 units above the midpoint, and ending 2 units below
    %% the midpoint.
    \draw [dashed] ($(midpoint) + (0, 2.2)$) -- ($(midpoint) + (0, -2.2)$);
  \end{feynman}
\end{tikzpicture}
\end{aligned}	
\end{equation}
&
\vspace{-1cm}

\begin{equation}	\tag{\text{dcut B}} 
    \begin{aligned}
\tikzfeynmanset{ whiteblob/.style={ shape=circle, typeset=$1L$,
draw=black, } }
\tikzfeynmanset{ myblob/.style={ shape=circle, typeset=$tree$,
draw=black, } }
\begin{tikzpicture}
  \begin{feynman}
    \diagram [ horizontal=b to c] {
      b [myblob] --  [white] db -- [white] c [whiteblob], %uso solo per distanziare i due blob, ma essendo bianchi verranno ricoperti
      b -- [white] ds -- [white] c,
      a [particle=\(3^+\)] -- [gluon] b
        -- [gluon, half left, out=60, in=120, rmomentum=\(\ell_1\)] c
        -- [gluon, half left, in=120, out=60, rmomentum=\(\ell_2\)] b ,
      d1 [particle=\(2^+\)] -- [gluon] b,
      c -- [scalar] d [particle=\(\phi\)],
      c -- [gluon] d2 [particle=\(1^+\)],
    };
	\label{sphi1}
    %% Find the midpoint, which is halfway between b and c.
    \coordinate (midpoint) at ($(b)!0.5!(c)$);
    %% Draw a line starting 2 units above the midpoint, and ending 2 units below
    %% the midpoint.
    \draw [dashed] ($(midpoint) + (0, 2.1)$) -- ($(midpoint) + (0, -2.1)$);
  \end{feynman}
\end{tikzpicture}
\end{aligned}
\end{equation}
\end{tabularx}
In $s_\phi$-channel, the integrand for the double cut computation is
\begin{align*}
	A^{2L}_{int}|_{\text{dcut A}}&=-2  A^{tree}(\phi^\dagger;\ell_1^+,(-\ell_2)^+)\frac{\langle \ell_1 \ell_2 \rangle^3}{\langle 12 \rangle \langle 23 \rangle \langle 3 \ell_1 \rangle \langle \ell_2 1 \rangle}\\
	&=A^{1L}(\phi;1^+,2^+,3^+)\frac{\tr_-(13\ell_1 \ell_2)}{\langle 3 \ell_1 3]\langle 1 \ell_2 1]}
\end{align*}
\begin{align*}
	\int \dd \Phi_2 A^{2L}_{int}|_{\text{dcut A}}&=A^{1L}(\phi;1^+,2^+,3^+) \left[\frac{1}{2}(-s_{1\phi} s_{3\phi}) I_4^{1m}(s_{1\phi},s_{3\phi};s_\phi)|_{s_\phi\text{-cut}}\right.\\
	&\hspace{0.5cm}\left.+ \frac{1}{2}(s_{1\phi}-s_\phi)I_3^{2m}(s_{1\phi},s_\phi)|_{s_\phi\text{-cut}}+ \frac{1}{2}(s_{3\phi}-s_{\phi})I_3^{2m}(s_{3\phi},s_\phi)|_{s_\phi\text{-cut}}\right].
\end{align*}
In $s_{\phi1}$-channel the double cut can be easily computed using Schouten identity,
\begin{align*}
	A^{2L}_{int}|_{\text{dcut B}}&=-2 s_\phi^2 A^{tree}(\phi^\dagger;1^+,\ell_1^+,(-\ell_2)^+)\frac{\langle \ell_1 \ell_2 \rangle^3}{\langle 23 \rangle \langle 3 \ell_1 \rangle \langle \ell_2 2 \rangle}\\
	&=A^{1L}(\phi;1^+,2^+,3^+)\left(\frac{\langle 1 \ell_2 \rangle}{\langle \ell_1 1 \rangle}+\frac{\langle \ell_2 3 \rangle}{\langle \ell_1 3 \rangle}\right)\left(\frac{\langle 1 \ell_1 \rangle}{\langle \ell_2 1 \rangle}+\frac{\langle \ell_1 2 \rangle}{\langle \ell_2 2 \rangle}\right)\\
	&=A^{1L}(\phi;1^+,2^+,3^+)\left[\frac{1}{2}(-s_{1\phi}s_{3\phi})I_4^{1m}(s_{1\phi},s_{3\phi};s_\phi)|_{s_{\phi1}\text{-cut}}+\right.\\
	&\hspace{0.5cm}\left.\frac{1}{2}(-s_{1\phi}s_{2\phi})I_4^{1m}(s_{1\phi},s_{2\phi};s_\phi)|_{s_{\phi1}\text{-cut}} +2\frac{1}{2}(s_{1\phi}-s_\phi) I_3^{2m}(s_{1\phi},s_\phi)|_{s_{\phi1}\text{-cut}}\right]
\end{align*}
The self-dual Higgs is unordered and this is the motivation of the presence of two boxes with a different configuration of gluons.
We do not observe the presence of contributions in this channel that cannot be investigated in the previous one. Hence, we only have one-mass boxes and two-mass triangles with $\phi$ alone.
\section{Cuts with a 1L YM sub-amplitude}
\begin{tabularx}{\linewidth}{XX}
\vspace{-1cm}
\begin{equation}	\tag{\text{dcut C}} 
    \begin{aligned}
\tikzfeynmanset{ myblob/.style={ shape=circle, typeset=$1L$,
draw=black, } }
\tikzfeynmanset{ whiteblob/.style={ shape=circle, typeset=$tree$,
draw=black, } }
\begin{tikzpicture}
  \begin{feynman}
    \diagram [ horizontal=b to c] {
      b [myblob] --  [white] db -- [white] c [whiteblob], %uso solo per distanziare i due blob, ma essendo bianchi verranno ricoperti
      b -- [white] ds -- [white] c,
      a [particle=\(3^+\)] -- [gluon] b
        -- [gluon, half left, out=60, in=120, rmomentum=\(\ell_1\)] c
        -- [gluon, half left, in=120, out=60, rmomentum=\(\ell_2\)] b ,
      d1 [particle=\(2^+\)] -- [gluon] b,
      c -- [scalar] d [particle=\(\phi\)],
      c -- [gluon] d2 [particle=\(1^+\)],
    };
	\label{sphi1}
    %% Find the midpoint, which is halfway between b and c.
    \coordinate (midpoint) at ($(b)!0.5!(c)$);
    %% Draw a line starting 2 units above the midpoint, and ending 2 units below
    %% the midpoint.
    \draw [dashed] ($(midpoint) + (0, 2.1)$) -- ($(midpoint) + (0, -2.1)$);
  \end{feynman}
\end{tikzpicture}
\end{aligned}
\end{equation}
&
\vspace{-1cm}
\begin{equation}	\tag{\text{dcut D}} 
    \begin{aligned}
\tikzfeynmanset{ myblob/.style={ shape=circle, typeset=$1L$,
draw=black, } }
\tikzfeynmanset{ whiteblob/.style={ shape=circle, typeset=$tree$,
draw=black, } }
\begin{tikzpicture}
  \begin{feynman}
    \diagram [scale=0.9, horizontal=b to c] {
      b [myblob] --  [white] db -- [white] c [whiteblob], %uso solo per distanziare i due blob, ma essendo bianchi verranno ricoperti
      b -- [white] ds -- [white] c,
      a [particle=\(2^+\)] -- [gluon] b
        -- [gluon, half left, out=60, in=120, momentum=\(\ell_1\)] c
        -- [gluon, half left, in=120, out=60, momentum=\(\ell_2\)] b ,
      d1 [particle=\(1^+\)] -- [gluon] b,
      d3 [particle=\(3^+\)]-- [gluon] b,
      c -- [scalar] d [particle=\(\phi\)],
    };

    %% Find the midpoint, which is halfway between b and c.
    \coordinate (midpoint) at ($(b)!0.5!(c)$);
    %% Draw a line starting 2 units above the midpoint, and ending 2 units below
    %% the midpoint.
    \draw [dashed] ($(midpoint) + (0, 2.2)$) -- ($(midpoint) + (0, -2.2)$);
  \end{feynman}
\end{tikzpicture}
\end{aligned}	
\end{equation}
\end{tabularx}
In $s_{\phi 1}$-channel, the product of sub-amplitude is
\begin{align*}
	A^{2L}_{int}|_{\text{dcut C}}&=\left[\frac{\langle \ell_1 \ell_2 \rangle^3}{\langle 1 \ell_1 \rangle \langle \ell_2 1 \rangle}\right] \left[ \frac{d_s-2}{2}\frac{1}{3}\frac{[\ell_1\ell_2][23]}{-\langle \ell_1 \ell_2 \rangle\langle 23 \rangle}\right]\\
	&=-\frac{d_s-2}{2}\frac{1}{3}[23]^2 \frac{\langle \ell_1 \ell_2 \rangle}{\langle \ell_1 1 \rangle \langle 1 \ell_2 \rangle}=-\frac{d_s-2}{2}\frac{1}{3}[23]^2 \frac{1}{\langle 12 \rangle}\left(-\frac{\langle 2 \ell_2 1]}{\langle 1 \ell_2 1]}+\frac{\langle 2 \ell_2 1 ]}{\langle 1 \ell_1 1]}\right)\\
	&=-\frac{d_s-2}{2}\frac{1}{3}[23]^2 \frac{1}{\langle 12 \rangle}\left(-\frac{\langle 2 P_{12} 1]}{\langle 1 P_{12} 1]}+\frac{\langle 2 P_{12} 1 ]}{\langle 1 P_{12} 1]}\right)=0.
\end{align*}
The last passage can be shown using the explicit integrand reduction of 3-pts tensor integrals.\\

\noindent The double cut in $s_\phi$-channel was computed in the attached Mathematica notebook. We obtain only a bubble. Because of the colorless of $\phi$, we sum over the three possible configurations of gluons and we obtain a coefficient proportional to the one-loop amplitude,
$$
	-\frac{d_s-2}{6} A^{1L}(\phi;1^+,2^+,3^+) I_2(s_\phi).
$$
% Uncmment the following two lines if you want to have a bibliography
%\bibliographystyle{apalik%\bibliography{document}
\end{document}